\documentclass[12pt]{beamer}
%\usecolortheme{seagull}
%\usecolortheme{wolverine} yuk
%\usecolortheme{beetle}
\usecolortheme{dove} % black on white
\usepackage[T1]{fontenc}
\usepackage{garamond}
\usefonttheme{serif}
\usepackage{multicol}
\usepackage{pifont}
\usepackage{etex}
\usepackage{graphicx}
\usepackage{amsmath}
\usepackage{amsthm}
\usepackage{amssymb}
\usepackage{semantic}
\usepackage[all]{xy}
\usepackage{color}
\usepackage{listings}
\usepackage{fancybox}
\usepackage{stmaryrd}
\usepackage{rotating}
\usepackage{wasysym}
\usepackage{ulem}

\usepackage{enumitem}
\setitemize{label=\usebeamerfont*{itemize item}%
  \usebeamercolor[fg]{itemize item}
  \usebeamertemplate{itemize item}}
  \setlist{itemsep=1ex}



\newcommand{\Gbox}[1]{\colorbox{lightgray}{#1}}
\newcommand{\Rbox}[1]{\colorbox{pink}{#1}}

\newcommand{\featstart}{\hfill}
\newcommand{\featend}{\hfill\hfill}
\newcommand{\feat}[1]{{\featstart#1\featend}}

\newcommand{\Topcircle}{\begin{turn}{270}\Leftcircle\end{turn}}
\newcommand{\BOTTOMCIRCLE}{\begin{turn}{270}\RIGHTCIRCLE\end{turn}}
\newcommand{\halfcircle}{\parbox{0in}{\Topcircle}\parbox{1.65ex}{\BOTTOMCIRCLE}{}}

\newcommand{\featY}{\feat{\CIRCLE}} % Has feature fully
\newcommand{\featP}{\feat{\halfcircle}} % Has feature partially
\newcommand{\featN}{\feat{\Circle}} % Does not have feature


\newcommand{\labeltag}[1]{\label{#1}\tag{\textsc{#1}}}
\newcommand{\type}{\vdash}
\newcommand{\typeS}{\vdash_{STLC}}
\newcommand{\typeG}{\vdash}
\newcommand{\typeCC}{\vdash_{C}}

\newcommand{\evall}{\Downarrow }
\newcommand{\evallS}{\Downarrow_{STLC} }
\newcommand{\evallG}{\Downarrow}
\newcommand{\evallCC}{\Downarrow_{C}}
\newcommand{\evallD}{\Downarrow_{DTLC}}

\newcommand{\reduce}{\longrightarrow}
\newcommand{\becomes}{\longrightarrow}

\newcommand{\EE}{{\cal E}}
\newcommand{\FF}{{\cal F}}
\newcommand{\Hole}{\Box}

\newcommand{\divergeG}{\Uparrow}
\newcommand{\subtype}{<:}
\newcommand{\consis}{\sim}

\newcommand{\embed}[1]{\lceil #1 \rceil}
\newcommand{\bl}[1]{{\color{blue} #1}}
\newcommand{\rd}[1]{{\color{red} #1}}
\newcommand{\pr}[1]{{\color{purple} #1}}
\newcommand{\kw}[1]{\mathtt{#1}}

\newcommand{\labels}[1]{\mathit{labels}(#1)}
\newcommand{\static}[2]{\mathit{static}(#1,#2)}
\newcommand{\safe}[1]{\mathrel{\mathit{safe}} #1}
\newcommand{\lo}[1]{\overline{#1}}
\newcommand{\rng}[1]{\mathit{rng}(#1)}

\newcommand{\semi}{\mathbin{;}}
\newcommand{\id}{\key{id}}
\newcommand{\Id}[1]{\id_{#1}}
\newcommand{\fail}[3]{\bot^{#1}_{#2 \Rightarrow #3}}
\newcommand{\Fail}[1]{\bot^{#1}}
\newcommand{\FAIL}[3]{\bot^{#2}}
\newcommand{\qu}[2]{{{#2}\query^{#1}}}
\newcommand{\pl}[1]{{#1\pling}}
\newcommand{\query}{\mathtt{?}}
\newcommand{\pling}{\mathtt{!}}

\newcommand{\bcfun}[1]{\langle\!\langle #1 \rangle\!\rangle}
\newcommand{\MergeT}{\sqcap}
\newcommand{\RefC}[1]{\key{Ref}(#1)}
\newcommand{\error}{\key{error}}
\newcommand{\rtti}[2]{#1(#2)_{\mathsf{rtti}}}
\newcommand{\val}[2]{#1(#2)_{\mathsf{val}}}

\newcommand{\Obj}{\key{Obj}}
\newcommand{\String}{\key{String}}
\newcommand{\Double}{\key{Double}}

%\newcommand{\If}[3]{\key{if}\,#1\key{if}\,#2\key{if}#3}


\newcommand{\ba}{\begin{array}}
\newcommand{\ea}{\end{array}}
\newenvironment{stack}{\ba{@{}l@{}}}{\ea}
\newenvironment{branch}{\left\{\ba{@{}l@{\qquad}l@{}}}{\ea\right\}}
\newenvironment{syntax}{\[\ba{l@{\;\;}lcl}}{\ea\]}
\newcommand{\dotspace}{.\,}
\newcommand{\key}[1]{\ensuremath{\mathtt{#1}}}
\newcommand{\Base}{B}
\newcommand{\dyn}{\star}
\newcommand{\Dyn}{\ensuremath{\star}}
\newcommand{\Int}{\key{Int}}
\newcommand{\Nat}{\mathbb{N}}
\newcommand{\Float}{\key{float}}
\newcommand{\Bool}{\key{Bool}}
\newcommand{\Str}{\key{String}}
%\newcommand{\Ref}{\key{Ref}\,}
\newcommand{\lam}[1]{\lambda #1}
\newcommand{\Lam}[1]{\Lambda #1 \dotspace}
\newcommand{\by}{\mapsto}
\newcommand{\app}{\;\,}
\newcommand{\tapp}{\;\,}
\newcommand{\of}{{:}}
\newcommand{\tu}{{\to}}
\newcommand{\To}{\Rightarrow}
\newcommand{\Let}{\key{let}\;}
\newcommand{\Letrec}{\key{let}\,\key{rec}\;}
\newcommand{\In}{\key{in}\;}
\newcommand{\If}{\key{if}\;}
\newcommand{\Then}{\;\key{then}\;}
\newcommand{\Else}{\;\key{else}\;}
\newcommand{\True}{\key{true}}
\newcommand{\False}{\key{false}}
\newcommand{\zero}{\key{zero}}
\newcommand{\suc}[1]{\key{suc}(#1)}
\newcommand{\as}{\mathrel{\key{as}}}
\newcommand{\op}{\mathit{op}}
\newcommand{\dom}[1]{\mathit{dom}(#1)}
\newcommand{\cod}[1]{\mathit{cod}(#1)}
\newcommand{\blame}[1]{\key{blame}\,#1}
\newcommand{\pblame}[2]{\key{blame}\,#1@#2}
\newcommand{\ledyn}{\sqsubseteq}
\newcommand{\IS}{\mathrel{\mathtt{is}}}
\newcommand{\cast}[1]{\overset{#1}{\Rightarrow}}
%\newcommand{\mkcast}[1]{\langle\!\langle#1\rangle\!\rangle}
\newcommand{\mkcast}[1]{(#1)}
\newcommand{\alloc}{\key{ref}\,}
\newcommand{\deref}{\texttt{!}}
\newcommand{\update}{\mathrel{\texttt{:=}}}
\newcommand{\all}[1]{\forall #1.\,}
\newcommand{\ftv}[1]{\mathrm{ftv}(#1)}
\newcommand{\CAST}[1]{\langle #1 \rangle}
\newcommand{\new}[1]{\nu #1.\;}
\newcommand{\case}[3]{\key{case}\,#1\,#2\,#3}
\newcommand{\join}[2]{#1 \sqcup #2 }
\newcommand{\meet}[2]{#1 \sqcap #2 }

\newcommand*\oldmacro{}%
\let\oldmacro\insertshorttitle%
\renewcommand*\insertshorttitle{%
  \oldmacro\hfill%
  \insertframenumber\,/\,\inserttotalframenumber}

\setbeamertemplate{navigation symbols}{}
\setbeamertemplate{footline}[frame number]

%\newtheorem{definition}{Definition}
\newtheorem{conjecture}[theorem]{\translate{Conjecture}}
\newtheorem{proposition}[theorem]{\translate{Proposition}}

\lstdefinestyle{basic}{
%showstringspaces=false,
language=Python,
columns=fullflexible,
%basicstyle=\sffamily\small,%
basicstyle=\ttfamily,%
%columns=fixed,
%basewidth=0.49em,
%lineskip=0pt,
%escapechar=@,xleftmargin=1pc,%
keywordstyle=\ttfamily,
mathescape=true,%
moredelim=**[is][\color{blue}]{@}{@},
moredelim=[is][\color{red}]{|}{|},
moredelim=[is][\color{blue}]{~}{~},
%commentstyle=\rmfamily,%
%morekeywords={return,fix,var,proc,fun,func},%
%deletekeywords={int,bool}
}
\lstset{style=basic}

\garamond

\title[Type Safety via Step-Index Logic]{Intro. to Step-Indexed Logical Relations: Type Safety for STLC + fix}
\author{Jeremy G. Siek \\[1ex]
 Indiana University, Bloomington
}
\date{}
%% \institute{\normalsize 
%%  Indiana University, Bloomington
%% }

% 3 hours

%\newcommand\footnotemark{}
%\renewcommand\footnoterule{}
\setbeamercolor{footnote mark}{fg=white}

\begin{document}

%===============================================================================
\frame{
\titlepage

\vspace{-40pt}
\begin{tabular}{ccc}
\begin{minipage}{0.2\textwidth}
  %\includegraphics[height=2in]{knight}
  %image
  \ 
\end{minipage}
&
\begin{minipage}{0.45\textwidth}
\begin{center}
PL Wonks \\
November 2023 \\
\ \\
\ \\
\ \\
\ \\
\end{center}
\end{minipage}
&
\begin{minipage}{0.2\textwidth}
  %\includegraphics[height=2in]{peltast}
  %image
  \ 
\end{minipage}
\end{tabular}

}
%===============================================================================
\frame{
\frametitle{Outline}

\begin{itemize}
\item Review of
  \begin{itemize}
    \item STLC + fix
    \item Type Safety via Progress and Preservation
  \end{itemize}
\item The Logical Relations Recipe
\item Strawman Logical Relation for Type Safety
\item Step-indexed Logical Relation for Type Safety
\item Proof of the Fundamental Lemma
  \begin{itemize}  
  \item Proof of the Bind Lemma
  \item Proof of the Compatibility Lemmas
  \end{itemize}
\item Proof of Type Safety
\end{itemize}

}
%===============================================================================
\frame{
\frametitle{Review: STLC + fix}

\[
\begin{array}{llcl}
\text{types} & A,B &::=& \Nat \mid A \to B \\
\text{terms} & L,M,N &::= & \zero \mid \suc{M} \mid \case{L}{M}{N} \mid \\
  &&& i \mid \lam{N} \mid L \app N
      \mid \mu N \\
\text{values} & V,W & ::= & \zero \mid \suc{V} \mid \lam{N} \mid \mu V \\
\text{frames} & F & ::= & \suc{\Box} \mid \case{\Box}{M}{N}
  \mid \Box \app M \mid V \app \Box
\end{array}
\]
Substitution: $\qquad N[M]$ \\

Reduction
\begin{align*}
  (\mu V) \app W &\longrightarrow V[\mu V] \app W \\
  (\lam{N}) \app W &\longrightarrow N[W]\\
  \case{\zero}{M}{N} &\longrightarrow M \\
  \case{\suc{V}}{M}{N} &\longrightarrow N[V] \\
  F[M] &\longrightarrow F[N] & \text{if } M \longrightarrow N
\end{align*}

}
%===============================================================================
\frame{
\frametitle{Review: STLC + fix}

%% \[
%% \begin{array}{llcl}
%% \text{type env.} & \Gamma &::=& A_0, \ldots, A_n
%% \end{array}
%% \]

\fbox{$\Gamma \vdash^\mathcal{V} V : A$}
\begin{gather*}
  \inference{}
            {\Gamma \vdash^\mathcal{V} \zero : \Nat}
            \quad
  \inference{\Gamma \vdash^\mathcal{V} V : \Nat}
            {\Gamma \vdash^\mathcal{V} \suc{V} : \Nat}
            \\[2ex]
  \inference{\Gamma,A \vdash^\mathcal{V} N : B}{\Gamma \vdash^\mathcal{V} \lam{N} : A \to B}
  \inference{\Gamma,A \to B \vdash^\mathcal{V} V : A \to B}{\Gamma \vdash^\mathcal{V} \mu{V} : A \to B}
\end{gather*}

\fbox{$\Gamma \vdash M : A$}
\begin{gather*}
  \inference{\Gamma \vdash^{\mathcal{V}} V : A}{\Gamma \vdash V : A}
  \quad
  \inference{}
            {A_n,\ldots,A_0 \vdash i : A_i}(0 \le i \le n)
            \\[2ex]
  \inference{\Gamma \vdash L : A \to B & \Gamma \vdash M : A}
            {\Gamma \vdash L \app M : B}
            \\[1ex]
  \inference{\Gamma \vdash M : \Nat}
            {\Gamma \vdash \suc{M} : \Nat}
  \quad
            \inference{\Gamma \vdash L : \Nat \\
              \Gamma \vdash M : A &
              \Gamma,\Nat \vdash N : A}
                      {\Gamma \vdash \case{L}{M}{N} : A}
\end{gather*}

}

%===============================================================================
\frame{
  \frametitle{Review: Type Safety via Progress \& Preservation}

  \begin{lemma}[Progress]
    If $\Gamma \vdash M : A$ then either
    $M$ is a value or
    $M \longrightarrow N$ for some $N$.
  \end{lemma}

  \begin{lemma}[Presevation]
    If $\Gamma \vdash M : A$ and $M \longrightarrow N$ then
    $\Gamma \vdash N : A$.
  \end{lemma}

  \begin{theorem}[Type Safety]
    If $\;\vdash M : A$ then either
    $M \longrightarrow^{*} V$ for some $V$
    or $M$ diverges.
  \end{theorem}

}

%===============================================================================
\frame{
  \frametitle{The Logical Relations Recipe}

  \begin{itemize}
  \item Define two relations that generalize the theorem you'd like to
    prove: one on types and values $\mathcal{V}(A,V)$
    and the other on types and terms $\mathcal{E}(A,M)$.
  \item Extend the $\mathcal{E}$ relation to open terms.
    \begin{align*}
      \mathcal{G}(\Gamma, \sigma) &=
        \mathcal{V}(A_i,\sigma_i) \text{ for all } A_i \in \Gamma \\
        \Gamma \models M : A &=
        \forall \gamma,
      \mathcal{G}(\Gamma,\sigma) \text{ implies }
      \mathcal{E}(A, \sigma(M))
    \end{align*}
    
  \item Prove the Fundamental Lemma, that\\
    (1) $\Gamma \vdash^{\mathcal{V}} V : A$ implies $\Gamma \models V : A$ and\\
    (2) $\Gamma \vdash M : A$ implies $\Gamma \models M : A$.

  \item Prove that $\mathcal{E}(A,M)$ implies your theorem.
    
  \end{itemize}
}  

%===============================================================================
\end{document}
