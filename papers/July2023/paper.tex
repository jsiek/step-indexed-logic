%%
%% This is file `sample-acmsmall.tex',
%% generated with the docstrip utility.
%%
%% The original source files were:
%%
%% samples.dtx  (with options: `acmsmall')
%% 
%% IMPORTANT NOTICE:
%% 
%% For the copyright see the source file.
%% 
%% Any modified versions of this file must be renamed
%% with new filenames distinct from sample-acmsmall.tex.
%% 
%% For distribution of the original source see the terms
%% for copying and modification in the file samples.dtx.
%% 
%% This generated file may be distributed as long as the
%% original source files, as listed above, are part of the
%% same distribution. (The sources need not necessarily be
%% in the same archive or directory.)
%%
%%
%% Commands for TeXCount
%TC:macro \cite [option:text,text]
%TC:macro \citep [option:text,text]
%TC:macro \citet [option:text,text]
%TC:envir table 0 1
%TC:envir table* 0 1
%TC:envir tabular [ignore] word
%TC:envir displaymath 0 word
%TC:envir math 0 word
%TC:envir comment 0 0
%%
%%
%% The first command in your LaTeX source must be the \documentclass
%% command.
%%
%% For submission and review of your manuscript please change the
%% command to \documentclass[manuscript, screen, review]{acmart}.
%%
%% When submitting camera ready or to TAPS, please change the command
%% to \documentclass[sigconf]{acmart} or whichever template is required
%% for your publication.
%%
%%
\documentclass[acmsmall]{acmart}
\usepackage{semantic}
\usepackage{stmaryrd}
\usepackage{agda}
\usepackage{MnSymbol}


%%
%% \BibTeX command to typeset BibTeX logo in the docs
\AtBeginDocument{%
  \providecommand\BibTeX{{%
    Bib\TeX}}}

%% Rights management information.  This information is sent to you
%% when you complete the rights form.  These commands have SAMPLE
%% values in them; it is your responsibility as an author to replace
%% the commands and values with those provided to you when you
%% complete the rights form.
\setcopyright{acmcopyright}
\copyrightyear{2023}
\acmYear{2023}
\acmDOI{XXXXXXX.XXXXXXX}


%%
%% These commands are for a JOURNAL article.
%% \acmJournal{JACM}
%% \acmVolume{37}
%% \acmNumber{4}
%% \acmArticle{111}
%% \acmMonth{8}

%%
%% Submission ID.
%% Use this when submitting an article to a sponsored event. You'll
%% receive a unique submission ID from the organizers
%% of the event, and this ID should be used as the parameter to this command.
%%\acmSubmissionID{123-A56-BU3}

%%
%% For managing citations, it is recommended to use bibliography
%% files in BibTeX format.
%%
%% You can then either use BibTeX with the ACM-Reference-Format style,
%% or BibLaTeX with the acmnumeric or acmauthoryear sytles, that include
%% support for advanced citation of software artefact from the
%% biblatex-software package, also separately available on CTAN.
%%
%% Look at the sample-*-biblatex.tex files for templates showcasing
%% the biblatex styles.
%%

%%
%% The majority of ACM publications use numbered citations and
%% references.  The command \citestyle{authoryear} switches to the
%% "author year" style.
%%
%% If you are preparing content for an event
%% sponsored by ACM SIGGRAPH, you must use the "author year" style of
%% citations and references.
%% Uncommenting
%% the next command will enable that style.
\citestyle{acmauthoryear}

\newcommand{\sem}[1]{\llbracket #1 \rrbracket}
\newcommand{\semV}[1]{\mathcal{V}\sem{#1}}
\newcommand{\semE}[1]{\mathcal{E}\sem{#1}}
\newcommand{\ba}{\begin{array}}
\newcommand{\ea}{\end{array}}
\newenvironment{stack}{\ba{@{}l@{}}}{\ea}
\newenvironment{branch}{\left\{\ba{@{}l@{\qquad}l@{}}}{\ea\right\}}
\newenvironment{syntax}{\[\ba{l@{\;\;}lcl}}{\ea\]}
\newcommand{\dotspace}{.\,}
\newcommand{\key}[1]{\ensuremath{\mathtt{#1}}}
\newcommand{\dyn}{\star}
\newcommand{\Dyn}{\ensuremath{\dyn}}
\newcommand{\Int}{\key{int}}
\newcommand{\Bool}{\key{bool}}
\newcommand{\lam}[1]{\lambda #1 \dotspace}
\newcommand{\app}{\,}
\newcommand{\reduce}{\longrightarrow}
\newcommand{\inj}[2]{#1 \,!\, #2}
\newcommand{\proj}[2]{#1 \,?\, #2}
% \newcommand{\kapprox}[2]{\mathrm{approx}(#1, #2)}
\newcommand{\kapprox}[2]{↓_{#1}\, #2}
\newcommand{\varnow}{\mathsf{var}\mbox{-}\mathsf{now}}
\newcommand{\laters}{\mathsf{laters}}
\newcommand{\Now}{\mathsf{Now}}
\newcommand{\Later}{\mathsf{Later}}
\newcommand{\zero}{\mathsf{zero}}
\newcommand{\suc}[1]{\mathsf{suc}\app#1}
\newcommand{\eff}[1]{\Lbag#1\Rbag}
\newcommand{\AgdaMissingDefinition}[1]{#1}

\usepackage{newunicodechar}
\newunicodechar{←}{\ensuremath{\mathnormal\from}}
\newunicodechar{→}{\ensuremath{\mathnormal\to}}
\newunicodechar{↓}{\ensuremath{\mathnormal\downarrow}}
\newunicodechar{⇑}{\ensuremath{\mathnormal\Uparrow}}
\newunicodechar{⇓}{\ensuremath{\mathnormal\Downarrow}}
\newunicodechar{⇒}{\ensuremath{\Rightarrow}}
\newunicodechar{⇔}{\ensuremath{\iff}}
\newunicodechar{↠}{\ensuremath{\rightarrow^{\ast}}}
\newunicodechar{∀}{\ensuremath{\mathnormal\forall}}
\newunicodechar{∃}{\ensuremath{\exists}}
\newunicodechar{Π}{\ensuremath{\Pi}}
\newunicodechar{⊢}{\ensuremath{\vdash}}
\newunicodechar{⊣}{\ensuremath{\dashv}}
\newunicodechar{⊧}{\ensuremath{\models}}
\newunicodechar{⊨}{\ensuremath{\models}}
\newunicodechar{↦}{\ensuremath{\mapsto}}
\newunicodechar{⊥}{\ensuremath{\mathnormal\bot}}
\newunicodechar{⊤}{\ensuremath{\mathnormal\top}}
\newunicodechar{∷}{\ensuremath{\mathnormal::}}
\newunicodechar{⦂}{\ensuremath{:}}  % !!
\newunicodechar{ℕ}{\ensuremath{\mathbb N}}
\newunicodechar{ℤ}{\ensuremath{\mathbb Z}}
\newunicodechar{𝒫}{\ensuremath{\mathcal P}}
\newunicodechar{≤}{\ensuremath{\leq}}
\newunicodechar{×}{\ensuremath{\times}}
\newunicodechar{∸}{\ensuremath{\minus}}
\newunicodechar{′}{\ensuremath{'}}
\newunicodechar{₁}{\ensuremath{_1}}
\newunicodechar{₂}{\ensuremath{_2}}
\newunicodechar{₃}{\ensuremath{_3}}
\newunicodechar{₄}{\ensuremath{_4}}
\newunicodechar{₅}{\ensuremath{_5}}
\newunicodechar{ⱼ}{\ensuremath{_j}}
\newunicodechar{⊎}{\ensuremath{\cupdot}}
\newunicodechar{∪}{\ensuremath{\cup}}
\newunicodechar{ᵖ}{\ensuremath{^p}}
\newunicodechar{≡}{\ensuremath{\equiv}}
\newunicodechar{≢}{\ensuremath{\not\equiv}}
\newunicodechar{∘}{\ensuremath{\circ}}
\newunicodechar{·}{\ensuremath{\cdot}}
\newunicodechar{ᵒ}{\ensuremath{^\circ}}
\newunicodechar{″}{\ensuremath{''}}
\newunicodechar{▷}{\ensuremath{\rhd}}
\newunicodechar{◁}{\ensuremath{\lhd}}
\newunicodechar{∈}{\ensuremath{\in}}
\newunicodechar{⩦}{\ensuremath{\doteq}} % !!
\newunicodechar{∋}{\ensuremath{\ni}}
\newunicodechar{ˢ}{\ensuremath{^s}}
\newunicodechar{ᵈ}{\ensuremath{^d}}
\newunicodechar{Γ}{\ensuremath{\Gamma}}
\newunicodechar{∅}{\ensuremath{\emptyset}}
\newunicodechar{⟨}{\ensuremath{\langle}}
\newunicodechar{⟩}{\ensuremath{\rangle}}
\newunicodechar{⟪}{\ensuremath{\langle\!\!\langle}}
\newunicodechar{⟫}{\ensuremath{\rangle\!\!\rangle}}
\newunicodechar{⟅}{\ensuremath{\Lbag}}
\newunicodechar{⟆}{\ensuremath{\Rbag}}
               
\newunicodechar{α}{\ensuremath{\alpha}}
\newunicodechar{β}{\ensuremath{\beta}}
\newunicodechar{γ}{\ensuremath{\gamma}}
\newunicodechar{δ}{\ensuremath{\delta}}
\newunicodechar{Δ}{\ensuremath{\Delta}}
\newunicodechar{ξ}{\ensuremath{\xi}}
\newunicodechar{λ}{\ensuremath{\mathnormal\lambda}}
\newunicodechar{ƛ}{\ensuremath{\mathnormal\lambdabar}}
\newunicodechar{Λ}{\ensuremath{\mathnormal\Lambda}}
\newunicodechar{μ}{\ensuremath{\mu}}
\newunicodechar{π}{\ensuremath{\pi}}
\newunicodechar{ν}{\ensuremath{\nu}}
\newunicodechar{ϕ}{\ensuremath{\phi}}
\newunicodechar{ψ}{\ensuremath{\psi}}
\newunicodechar{þ}{\ensuremath{\chi}} % !!
\newunicodechar{σ}{\ensuremath{\sigma}}
\newunicodechar{Σ}{\ensuremath{\Sigma}}
\newunicodechar{τ}{\ensuremath{\tau}}
\newunicodechar{♯}{\ensuremath{\sharp}}
\newunicodechar{♭}{\ensuremath{\flat}}
\newunicodechar{ℓ}{\ensuremath{\ell}}
\newunicodechar{ₒ}{\ensuremath{_o}}
\newunicodechar{ₛ}{\ensuremath{_s}}
\newunicodechar{ₚ}{\ensuremath{_p}}
\newunicodechar{ₐ}{\ensuremath{_a}}
\newunicodechar{ᵃ}{\ensuremath{^a}}
\newunicodechar{ⁱ}{\ensuremath{^i}}
\newunicodechar{ⱽ}{\ensuremath{^V}}
\newunicodechar{∎}{\ensuremath{\blacksquare}} % !!
\newunicodechar{■}{\ensuremath{\blacksquare}}
\newunicodechar{□}{\ensuremath{\Box}}
\newunicodechar{•}{\ensuremath{\bullet}}
\newunicodechar{⦅}{\ensuremath{\llparenthesis}}
\newunicodechar{⦆}{\ensuremath{\rrparenthesis}}
\newunicodechar{⟦}{\ensuremath{\llbracket}}
\newunicodechar{⟧}{\ensuremath{\rrbracket}}
\newunicodechar{⦉}{\ensuremath{\llparenthesis}} % Z notation...
\newunicodechar{⦊}{\ensuremath{\rrparenthesis}} % Z notation...
\newunicodechar{ℰ}{\ensuremath{\mathcal{E}}}
\newunicodechar{𝒱}{\ensuremath{\mathcal{V}}}
\newunicodechar{𝓖}{\ensuremath{\mathcal{G}}}
\newunicodechar{𝓔}{\ensuremath{\mathcal{E}}}
\newunicodechar{⨟}{\ensuremath{\fatsemi}}
\newunicodechar{≐}{\ensuremath{\doteq}}
\newunicodechar{⊂}{\ensuremath{\subset}}
\newunicodechar{⌊}{\ensuremath{\lfloor}}
\newunicodechar{⌋}{\ensuremath{\rfloor}}
\newunicodechar{ᵇ}{\ensuremath{^b}}
\newunicodechar{ᵏ}{\ensuremath{^k}}
\newunicodechar{◇}{\ensuremath{\diamond}}
\newunicodechar{ˡ}{\ensuremath{^l}}


                 
%%
%% end of the preamble, start of the body of the document source.
\begin{document}

%%
%% The "title" command has an optional parameter,
%% allowing the author to define a "short title" to be used in page headers.
\title{A Step-Indexed Logic in Agda}

%%
%% The "author" command and its associated commands are used to define
%% the authors and their affiliations.
%% Of note is the shared affiliation of the first two authors, and the
%% "authornote" and "authornotemark" commands
%% used to denote shared contribution to the research.
\author{Jeremy G. Siek}
%\authornote{Both authors contributed equally to this research.}
\email{jsiek@indiana.edu}
\orcid{0000-0002-9894-4856}
%% \author{G.K.M. Tobin}
%% \authornotemark[1]
%% \email{webmaster@marysville-ohio.com}
\affiliation{%
  \institution{Indiana University}
  \streetaddress{Luddy Hall, 700 N. Woodlawn Avenue}
  \city{Bloomington}
  \state{IN}
  \country{USA}
  \postcode{47408}
}

\author{Philip Wadler}
\affiliation{%
  \institution{University of Edinburgh}
  %% \streetaddress{1 Th{\o}rv{\"a}ld Circle}
  \city{Edinburgh}
  \country{Scotland}}
%% \email{larst@affiliation.org}

\author{Peter Thiemann}
\affiliation{%
  \institution{University of Freiburg}
  \city{Freiburg}
  \country{Germany}
}

%%
%% By default, the full list of authors will be used in the page
%% headers. Often, this list is too long, and will overlap
%% other information printed in the page headers. This command allows
%% the author to define a more concise list
%% of authors' names for this purpose.
%\renewcommand{\shortauthors}{Trovato et al.}

%
%  step-indexed logics 
%
%
%



%%
%% The abstract is a short summary of the work to be presented in the
%% article.
\begin{abstract}
  Step-indexed logical relations are an important tool in the
  metatheory of programming
  languages~\citep{Appel:2001aa,Ahmed:2006aa}, and their application
  can be streamlined by use of step-indexed modal logics such as
  LSLR~\citep{Dreyer:2011wl}.  This paper describes a mechanization of
  a step-indexed modal logic in the Agda proof assistant and its
  application to a semantic type safety proof for the Simply-Typed
  Lambda Calculus extended with recursive functions. The step-indexed
  logic supports guarded recursive predicates via the later modality
  and provides a fixpoint theorem adapted from \citet{Appel:2001aa}.
\end{abstract}


%%
%% The code below is generated by the tool at http://dl.acm.org/ccs.cfm.
%% Please copy and paste the code instead of the example below.
%%
%\begin{CCSXML}
%\end{CCSXML}

%% \ccsdesc[500]{Computer systems organization~Embedded systems}
%% \ccsdesc[300]{Computer systems organization~Redundancy}
%% \ccsdesc{Computer systems organization~Robotics}
%% \ccsdesc[100]{Networks~Network reliability}

%%
%% Keywords. The author(s) should pick words that accurately describe
%% the work being presented. Separate the keywords with commas.
\keywords{step-indexed logical relations, modal logic}

%% \received{20 February 2007}
%% \received[revised]{12 March 2009}
%% \received[accepted]{5 June 2009}

%%
%% This command processes the author and affiliation and title
%% information and builds the first part of the formatted document.
\maketitle

%TODO: Road Map/Outline of what's in the paper

\clearpage

\tableofcontents

\section{Overview}
\label{sec:intro}

Logical relations are a useful technique for proving properties of
programming languages, from type safety~\citep{Timany:2022aa} to
relational parametricity~\citep{REYNOLDS83}, and
noninterference~\citep{heintze1998slam}.  Initially, logical relations
were applied to terminating languages such as the Simply Typed Lambda
Calculus~\citep{Tait:1967aa} and System F~\citep{GIRARD72}.
\citet{Appel:2001aa} introduced step-indexed logical relations to
handle languages with recursive types.  \citet{Dreyer:2011wl} designed
the LSLR logic to streamline the reasoning about step indexed logical
relations.  This work eventually led to Iris~\citep{JUNG:2018aa}, a
system that supports a combination of step indexing and separation
logic within the Coq proof
assistant~\citep{The-Coq-Development-Team:2004kf,Huet:2016aa}.

The authors are interested in using step-indexed logical relations for
purposes of developing mechanized proofs in Agda of the Gradual
Guarantee~\citep{Siek:2015ac} and Parametricity~\citep{REYNOLDS74C}
for gradually-typed polymorphic
languages~\citep{Ahmed:2011fk,Ahmed:2017aa,Igarashi:2017aa,New:2019ab,Labrada:2020tk}. One
option would be to use Guarded Cubical Agda~\citep{Veltri:2020aa}, but
both Cubical Agda~\citep{Vezzosi:2019aa} and its extension with
guarded fixpoints are experimental features of Agda that may not be
supported indefinitely.  We would prefer to have a step-indexed logic
that is embedded in (vanilla) Agda.

This paper fills that need by showing how to build a step-indexed
logic in Agda, a system that we abbreviate as SIL.  This paper is a
literate Agda file; all of the definitions and proofs have been
checked with Agda 2.6.2.2. 


%% Our goal is to define an operator for recursive predicates and relations
%% with syntax that is something like $μᵒ x. R$, where $x$ is the name of the
%% recursive relation and $R$ is the definition of the relation, which
%% can refer to $x$. We shall prove a fixpoint theorem which states that
%% the recursive predicate is equal to its unfolding, something like the
%% following.
%% \[
%%   (μᵒ x. R) \app δ \app a ≡ᵒ R \app δ(x:= μᵒ x. R) \app a
%% \]
%% where $δ$ is an environment mapping variables to predicates.

UNDER CONSTRUCTION

To evaluate the utility of SIL, Section~\ref{sec:STLC} presents a case
study of proving semantic type safety for the Simply Typed Lambda
Calculus extended with recursive functions. We find that SIL
streamlines the definition of logical relations and is helpful in many
of the proofs, but there is room for improvement regarding the
ergonomics of interactive proof development using SIL.  Limitations in
Agda's inference algorithm sometimes leads to verbosity in SIL proofs.


\input{latex/July2023/SILIntro.tex}

\input{latex/July2023/STLC2.tex}
\input{latex/July2023/LogRel2.tex}
\input{latex/July2023/STLCDeterministic2.tex}
\input{latex/July2023/STLCBind2.tex}
\input{latex/July2023/STLCSafe2.tex}
\input{latex/July2023/STLCTypeSafe.tex}

\section{Constructing a Step-Indexed Logic}

Most of the challenge in defining a step-indexed logic is in the
definition of recursive predicates and their fixpoint theorem. The
step-indexed models of \citet{Appel:2001aa} rely on an approximation
operator, here written as $↓_n$, to define the semantic criteria for
valid recursive definitions.  For example, they define a nonexpansive
function to be one that satisfies
\[
  ↓_n f(x) = ↓_n f(↓_n x)
\]
The step-indexed models of \citet{JUNG:2018aa} instead rely on
a step-indexed notion of equality. For example, they define a
nonexpansive function as one that satisfies
\[
  x \stackrel{n}{=} y \text{ implies } f(x) \stackrel{n}{=} f(y)
\]
We are not aware of any inherent reason to prefer one approach to the
other, so we choose the approach of \citet{Appel:2001aa} because they
provide perspicuous proofs of their fixpoint theorem.  So our approach
to SIL can be seen as an implementation of the LSLR logic of
\citet{Dreyer:2011wl} using the foundations of
\citet{Appel:2001aa}. In particular, SIL defines the recursive
predicate of SIL using function iteration, like this:
\[
  ⟦ μ F ⟧ \, x \, n = F^{n\plus 1}\, \top \, x \, n
\]
whereas LSLR uses substitution in the semantics of recursive types:
\[
   ⟦ μ r. R ⟧ \, x \, n = ⟦ R[μr.R/r] ⟧\,x\,n
\]

However, adapting the work of \citet{Appel:2001aa} to our setting was
nontrivial because (1) their model is for recursive types, not a modal
logic, and (2) their proofs were incomplete (which we learned part way
through our development of SIL). To enable the nesting of recursive
types inside other recursive types, one must prove that the $μ$ type
constructor is nonexpansive and well founded, but \citet{Appel:2001aa}
does not prove this (they prove those properties for the other type
constructors).  We discuss this problem and its remedy in sections
\ref{sec:appel-mcallester} and \ref{sec:open-propositions}.

We organize the rest of the article as follows. We review Appel an
McAllester's work in Section~\ref{sec:appel-mcallester}, then
introduce our representation of step-indexed propositions and
predicates in Agda in Sections \ref{sec:propositions} and
\ref{sec:predicates}.  We formalize several important concepts in
Section~\ref{sec:fun-approx-iter}: functionals, approximation, and
iteration, and prove some basic results about them.  We construct all
of the logical connectives of SIL in
Section~\ref{sec:open-propositions}, including the definition of
recursive predicates and their fixpoint theorem.  The proof theory of
SIL is developed in Section~\ref{sec:proofs}, where we construct the
proof rules of this modal logic.


\section{Review of Appel and McAllester's Fixpoint Theorem}
\label{sec:appel-mcallester}

Our proof of the fixpoint theorem for SIL is inspired by the fixpoint
theorem of \citet{Appel:2001aa}. Their fixpoint theorem proves that a
recursive type is equal to its unfolding.  They define a (semantic)
type $\tau$ to be a relation between step indexes and syntactic
values. They do not define a syntax for types, but instead define
operators for constructing semantic types as follows.
\begin{align*}
  ⊥ &= \{ \} \\
  ⊤ &= \{ ⟨k,v⟩ \mid k ∈ ℕ\} \\
  \mathbf{int} &= \{⟨k,n⟩ \mid k ∈ ℕ, n ∈ ℤ \}\\
  τ₁ × τ₂ &= \{ ⟨k,(v₁,v₂)⟩ \mid ∀j<k. ⟨j,v₁⟩∈τ₁, ⟨j,v₂⟩∈τ₂ \} \\
  τ₁ → τ₂ &= \{ ⟨k,λx.e⟩ \mid ∀j<k.∀v. ⟨j,v⟩∈τ₁ ⇒ e[v/x] :ⱼ τ₂ \} \\
  μ F &= \{ ⟨k,v⟩ \mid ⟨k,v⟩ ∈ F^{k\plus 1}(⊥) \}
    & \text{if } F : \tau \to \tau'
\end{align*}
Their fixpoint theorem says that for any well founded $F$,
\[
  μ F = F(μF)
\]
A well founded function $F$ on types is
one in which each pair in the output $⟨k,v⟩$ only depends
on later pairs in the input, that is, pairs of the form $⟨j,v′⟩$
where $j < k$. \citet{Appel:2001aa} characterize this dependency
with the help of the $k$-approximation function:
\[
  \kapprox{k}{τ} = \{ ⟨j,v⟩ \mid j < k, ⟨j,v⟩ ∈ τ\} 
\]
They define a \emph{well founded} function $F$ to be one that
satisfies the following equation.
\[
  \kapprox{k \plus 1}{F(τ)} = \kapprox{k \plus 1}{F(\kapprox{k}{τ})}
\]

Functions over semantic types are not always well founded.  For
example, the identity function $λα.α$ is not well founded, so one
cannot apply the fixpoint theorem to the recursive type $μ(λα.α)$
(which corresponds to the syntactic type $μα.α$).
On the other hand, the function
$λα.α×α$ is well founded because of the strict less-than in the
definition of the $×$ operator. So the fixpont theorem applies to
$μ(λα.α×α)$.  In general, a function built from the type operators is
well founded so long as the recursive $α$ only appears underneath a
type constructor such as $×$ or $→$. \citet{Appel:2001aa} prove this
fact, which relies on the auxilliary notion of a nonexpansive
function. In such a function, the output can depend on pairs at the
current step index as well as later ones. So a \emph{nonexpansive}
function satisfies the following equation.
\[
  \kapprox{k}{F(τ)} = \kapprox{k}{F(\kapprox{k}{τ})}
\]
For example, $λα.α$ is nonexpansive and so is $λα.\mathbf{int}$.
\citet{Appel:2001aa} then prove lemmas about the type constructors.
For example, regarding products, they prove that if $F$ and $G$
are nonexpansive functions, then so is $λ α. (F α) × (G α)$.

However, \citet{Appel:2001aa} neglect to prove such lemmas for the $μ$
operator itself. For example, given $F : τ₁ → τ₂ → τ₃$ that is
nonexpansive in its first parameter and well founded in its second,
then $λ α. μ (F α)$ is nonexpansive.  On the other hand, if $F$ is
well founded in both parameters, then $λ α. μ (F α)$ is well founded.

Comparing the type operators of \citet{Appel:2001aa} to the logic
operators of SIL, there are striking similarities. The function type
operator is quite similar to implication, although one difference is
that the function type operator uses strict less-than whereas
implication uses non-strict less-than. The logic introduces the later
operator, whereas the type operators essentially bake the later
operator into the type operators through their use of strict
less-than.

As noted above, our semantics of recursive predicates is similar to
the recursive type of \citet{Appel:2001aa} in that we define recursive
predicates via iteration.
%
However, we do not want to require users of the logic to have to prove
the well foundedness of their recursive formulas.  Instead, we
introduce a type system for propositions that ensure that $μ$ is only
applied to well founded formulas, and that the proof of well
foundedness is provided by our logical connectives, not by the user of
the logic.


\input{../../src/latex/StepIndexedLogic2.tex}
% \input{latex/July2023/SILReference.tex}


\bibliographystyle{ACM-Reference-Format}
\bibliography{all}

\appendix
\addcontentsline{toc}{section}{Appendix}
\section*{Appendix}


\end{document}
\endinput

