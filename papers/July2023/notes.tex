Most of the challenge in defining a step-indexed logic is in the
definition of recursive predicates and their fixpoint theorem. The
step-indexed models of \citet{Appel:2001aa} rely on an approximation
operator, here written as $↓_n$, to define the semantic criteria for
valid recursive definitions.  For example, they define a nonexpansive
function to be one that satisfies
\[
  ↓_n f(x) = ↓_n f(↓_n x)
\]
The step-indexed models of \citet{JUNG:2018aa} instead rely on
a step-indexed notion of equality. For example, they define a
nonexpansive function as one that satisfies
\[
  x \stackrel{n}{=} y \text{ implies } f(x) \stackrel{n}{=} f(y)
\]
We are not aware of any inherent reason to prefer one approach to the
other, so we choose the approach of \citet{Appel:2001aa} because they
provide perspicuous proofs of their fixpoint theorem.  So our approach
to SIL can be seen as an implementation of the LSLR logic of
\citet{Dreyer:2011wl} using the foundations of
\citet{Appel:2001aa}. In particular, SIL defines the recursive
predicate of SIL using function iteration, like this:
\[
  ⟦ μ F ⟧ \, x \, n = F^{n\plus 1}\, \top \, x \, n
\]
whereas LSLR uses substitution in the semantics of recursive types:
\[
   ⟦ μ r. R ⟧ \, x \, n = ⟦ R[μr.R/r] ⟧\,x\,n
\]

However, adapting the work of \citet{Appel:2001aa} to our setting was
nontrivial because (1) their model is for recursive types, not a modal
logic, and (2) their proofs were incomplete (which we learned part way
through our development of SIL). To enable the nesting of recursive
types inside other recursive types, one must prove that the $μ$ type
constructor is nonexpansive and well founded, but \citet{Appel:2001aa}
does not prove this (they prove those properties for the other type
constructors).  We discuss this problem and its remedy in sections
\ref{sec:appel-mcallester} and \ref{sec:open-propositions}.

We organize the rest of the article as follows. We review Appel an
McAllester's work in §\ref{sec:appel-mcallester}, then
introduce our representation of step-indexed propositions and
predicates in Agda in Sections \ref{sec:propositions} and
\ref{sec:predicates}.  We formalize several important concepts in
§\ref{sec:fun-approx-iter}: functionals, approximation, and
iteration, and prove some basic results about them.  We construct all
of the logical connectives of SIL in
§\ref{sec:open-propositions}, including the definition of
recursive predicates and their fixpoint theorem.  The proof theory of
SIL is developed in §\ref{sec:proofs}, where we construct the
proof rules of this modal logic.
