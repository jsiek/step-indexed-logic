\section{Review of Appel and McAllester's Fixpoint Theorem}
\label{sec:appel-mcallester}

Our proof of the fixpoint theorem for SIL is inspired by the fixpoint
theorem of \citet{Appel:2001aa}. Their fixpoint theorem proves that a
recursive type is equal to its unfolding.  They define a (semantic)
type $\tau$ to be a relation between step indexes and syntactic
values. They do not define a syntax for types, but instead define
operators for constructing semantic types as follows.
\begin{align*}
  ⊥ &= \{ \} \\
  ⊤ &= \{ ⟨k,v⟩ \mid k ∈ ℕ\} \\
  \mathbf{int} &= \{⟨k,n⟩ \mid k ∈ ℕ, n ∈ ℤ \}\\
  τ₁ × τ₂ &= \{ ⟨k,(v₁,v₂)⟩ \mid ∀j<k. ⟨j,v₁⟩∈τ₁, ⟨j,v₂⟩∈τ₂ \} \\
  τ₁ → τ₂ &= \{ ⟨k,λx.e⟩ \mid ∀j<k.∀v. ⟨j,v⟩∈τ₁ ⇒ e[v/x] :ⱼ τ₂ \} \\
  μ F &= \{ ⟨k,v⟩ \mid ⟨k,v⟩ ∈ F^{k\plus 1}(⊥) \}
    & \text{if } F : \tau \to \tau'
\end{align*}
Their fixpoint theorem says that for any well founded $F$,
\[
  μ F = F(μF)
\]
A well founded function $F$ on types is
one in which each pair in the output $⟨k,v⟩$ only depends
on later pairs in the input, that is, pairs of the form $⟨j,v′⟩$
where $j < k$. \citet{Appel:2001aa} characterize this dependency
with the help of the $k$-approximation function:
\[
  \kapprox{k}{τ} = \{ ⟨j,v⟩ \mid j < k, ⟨j,v⟩ ∈ τ\} 
\]
They define a \emph{well founded} function $F$ to be one that
satisfies the following equation.
\[
  \kapprox{k \plus 1}{F(τ)} = \kapprox{k \plus 1}{F(\kapprox{k}{τ})}
\]

Functions over semantic types are not always well founded.  For
example, the identity function $λα.α$ is not well founded, so one
cannot apply the fixpoint theorem to the recursive type $μ(λα.α)$
(which corresponds to the syntactic type $μα.α$).
On the other hand, the function
$λα.α×α$ is well founded because of the strict less-than in the
definition of the $×$ operator. So the fixpont theorem applies to
$μ(λα.α×α)$.  In general, a function built from the type operators is
well founded so long as the recursive $α$ only appears underneath a
type constructor such as $×$ or $→$. \citet{Appel:2001aa} prove this
fact, which relies on the auxilliary notion of a nonexpansive
function. In such a function, the output can depend on pairs at the
current step index as well as later ones. So a \emph{nonexpansive}
function satisfies the following equation.
\[
  \kapprox{k}{F(τ)} = \kapprox{k}{F(\kapprox{k}{τ})}
\]
For example, $λα.α$ is nonexpansive and so is $λα.\mathbf{int}$.
\citet{Appel:2001aa} then prove lemmas about the type constructors.
For example, regarding products, they prove that if $F$ and $G$
are nonexpansive functions, then so is $λ α. (F α) × (G α)$.

However, \citet{Appel:2001aa} neglect to prove such lemmas for the $μ$
operator itself. For example, given $F : τ₁ → τ₂ → τ₃$ that is
nonexpansive in its first parameter and well founded in its second,
then $λ α. μ (F α)$ is nonexpansive.  On the other hand, if $F$ is
well founded in both parameters, then $λ α. μ (F α)$ is well founded.

Comparing the type operators of \citet{Appel:2001aa} to the logic
operators of SIL, there are striking similarities. The function type
operator is quite similar to implication, although one difference is
that the function type operator uses strict less-than whereas
implication uses non-strict less-than. The logic introduces the later
operator, whereas the type operators essentially bake the later
operator into the type operators through their use of strict
less-than.

As noted above, our semantics of recursive predicates is similar to
the recursive type of \citet{Appel:2001aa} in that we define recursive
predicates via iteration.
%
However, we do not want to require users of the logic to have to prove
the well foundedness of their recursive formulas.  Instead, we
introduce a type system for propositions that ensure that $μ$ is only
applied to well founded formulas, and that the proof of well
foundedness is provided by our logical connectives, not by the user of
the logic.
