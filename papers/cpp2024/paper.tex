%%
%% This is file `sample-acmsmall.tex',
%% generated with the docstrip utility.
%%
%% The original source files were:
%%
%% samples.dtx  (with options: `acmsmall')
%% 
%% IMPORTANT NOTICE:
%% 
%% For the copyright see the source file.
%% 
%% Any modified versions of this file must be renamed
%% with new filenames distinct from sample-acmsmall.tex.
%% 
%% For distribution of the original source see the terms
%% for copying and modification in the file samples.dtx.
%% 
%% This generated file may be distributed as long as the
%% original source files, as listed above, are part of the
%% same distribution. (The sources need not necessarily be
%% in the same archive or directory.)
%%
%%
%% Commands for TeXCount
%TC:macro \cite [option:text,text]
%TC:macro \citep [option:text,text]
%TC:macro \citet [option:text,text]
%TC:envir table 0 1
%TC:envir table* 0 1
%TC:envir tabular [ignore] word
%TC:envir displaymath 0 word
%TC:envir math 0 word
%TC:envir comment 0 0
%%
%%
%% The first command in your LaTeX source must be the \documentclass
%% command.
%%
%% For submission and review of your manuscript please change the
%% command to \documentclass[manuscript, screen, review]{acmart}.
%%
%% When submitting camera ready or to TAPS, please change the command
%% to \documentclass[sigconf]{acmart} or whichever template is required
%% for your publication.
%%
%%
\documentclass[acmsmall]{acmart}
\usepackage{semantic}
\usepackage{agda}
\usepackage{MnSymbol}


%%
%% \BibTeX command to typeset BibTeX logo in the docs
\AtBeginDocument{%
  \providecommand\BibTeX{{%
    Bib\TeX}}}

%% Rights management information.  This information is sent to you
%% when you complete the rights form.  These commands have SAMPLE
%% values in them; it is your responsibility as an author to replace
%% the commands and values with those provided to you when you
%% complete the rights form.
\setcopyright{acmcopyright}
\copyrightyear{2023}
\acmYear{2023}
\acmDOI{XXXXXXX.XXXXXXX}


%%
%% These commands are for a JOURNAL article.
%% \acmJournal{JACM}
%% \acmVolume{37}
%% \acmNumber{4}
%% \acmArticle{111}
%% \acmMonth{8}

%%
%% Submission ID.
%% Use this when submitting an article to a sponsored event. You'll
%% receive a unique submission ID from the organizers
%% of the event, and this ID should be used as the parameter to this command.
%%\acmSubmissionID{123-A56-BU3}

%%
%% For managing citations, it is recommended to use bibliography
%% files in BibTeX format.
%%
%% You can then either use BibTeX with the ACM-Reference-Format style,
%% or BibLaTeX with the acmnumeric or acmauthoryear sytles, that include
%% support for advanced citation of software artefact from the
%% biblatex-software package, also separately available on CTAN.
%%
%% Look at the sample-*-biblatex.tex files for templates showcasing
%% the biblatex styles.
%%

%%
%% The majority of ACM publications use numbered citations and
%% references.  The command \citestyle{authoryear} switches to the
%% "author year" style.
%%
%% If you are preparing content for an event
%% sponsored by ACM SIGGRAPH, you must use the "author year" style of
%% citations and references.
%% Uncommenting
%% the next command will enable that style.
\citestyle{acmauthoryear}

\newcommand{\sem}[1]{\llbracket #1 \rrbracket}
\newcommand{\semV}[1]{\mathcal{V}\sem{#1}}
\newcommand{\semE}[1]{\mathcal{E}\sem{#1}}
\newcommand{\ba}{\begin{array}}
\newcommand{\ea}{\end{array}}
\newenvironment{stack}{\ba{@{}l@{}}}{\ea}
\newenvironment{branch}{\left\{\ba{@{}l@{\qquad}l@{}}}{\ea\right\}}
\newenvironment{syntax}{\[\ba{l@{\;\;}lcl}}{\ea\]}
\newcommand{\dotspace}{.\,}
\newcommand{\key}[1]{\ensuremath{\mathtt{#1}}}
\newcommand{\dyn}{\star}
\newcommand{\Dyn}{\ensuremath{\dyn}}
\newcommand{\Int}{\key{int}}
\newcommand{\Bool}{\key{bool}}
\newcommand{\lam}[1]{\lambda #1 \dotspace}
\newcommand{\app}{\,}
\newcommand{\reduce}{\longrightarrow}
\newcommand{\inj}[2]{#1 \,!\, #2}
\newcommand{\proj}[2]{#1 \,?\, #2}

\usepackage{newunicodechar}
\newunicodechar{λ}{\ensuremath{\mathnormal\lambda}}
\newunicodechar{←}{\ensuremath{\mathnormal\from}}
\newunicodechar{→}{\ensuremath{\mathnormal\to}}
\newunicodechar{↓}{\ensuremath{\mathnormal\downarrow}}
\newunicodechar{∀}{\ensuremath{\mathnormal\forall}}
\newunicodechar{∃}{\ensuremath{\exists}}
\newunicodechar{⊥}{\ensuremath{\mathnormal\bot}}
\newunicodechar{⊤}{\ensuremath{\mathnormal\top}}
\newunicodechar{∷}{\ensuremath{\mathnormal::}}
\newunicodechar{ℕ}{\ensuremath{\mathbb N}}
\newunicodechar{≤}{\ensuremath{\leq}}
\newunicodechar{∸}{\ensuremath{\div}}
\newunicodechar{′}{\ensuremath{'}}
\newunicodechar{⇒}{\ensuremath{\Rightarrow}}
\newunicodechar{₁}{\ensuremath{_1}}
\newunicodechar{₂}{\ensuremath{_2}}
\newunicodechar{Σ}{\ensuremath{\Sigma}}
\newunicodechar{⊎}{\ensuremath{\cupdot}}
\newunicodechar{ᵖ}{\ensuremath{^p}}
\newunicodechar{≡}{\ensuremath{\equiv}}
\newunicodechar{≢}{\ensuremath{\not\equiv}}
\newunicodechar{∘}{\ensuremath{\circ}}
\newunicodechar{ᵒ}{\ensuremath{^\circ}}
\newunicodechar{▷}{\ensuremath{\rhd}}
\newunicodechar{∈}{\ensuremath{\in}}
\newunicodechar{⇔}{\ensuremath{\iff}}
\newunicodechar{⩦}{\ensuremath{\equiv}} % !!
\newunicodechar{μ}{\ensuremath{\mu}}
\newunicodechar{∋}{\ensuremath{\ni}}
\newunicodechar{ˢ}{\ensuremath{^s}}
\newunicodechar{ᵈ}{\ensuremath{^d}}
\newunicodechar{Γ}{\ensuremath{\Gamma}}
\newunicodechar{∅}{\ensuremath{\emptyset}}
\newunicodechar{δ}{\ensuremath{\delta}}

%%
%% end of the preamble, start of the body of the document source.
\begin{document}

%%
%% The "title" command has an optional parameter,
%% allowing the author to define a "short title" to be used in page headers.
\title{A Step-Indexed Logic in Agda}

%%
%% The "author" command and its associated commands are used to define
%% the authors and their affiliations.
%% Of note is the shared affiliation of the first two authors, and the
%% "authornote" and "authornotemark" commands
%% used to denote shared contribution to the research.
\author{Jeremy G. Siek}
%\authornote{Both authors contributed equally to this research.}
\email{jsiek@indiana.edu}
\orcid{0000-0002-9894-4856}
%% \author{G.K.M. Tobin}
%% \authornotemark[1]
%% \email{webmaster@marysville-ohio.com}
\affiliation{%
  \institution{Indiana University}
  \streetaddress{Luddy Hall, 700 N. Woodlawn Avenue}
  \city{Bloomington}
  \state{IN}
  \country{USA}
  \postcode{47408}
}

%% \author{Lars Th{\o}rv{\"a}ld}
%% \affiliation{%
%%   \institution{The Th{\o}rv{\"a}ld Group}
%%   \streetaddress{1 Th{\o}rv{\"a}ld Circle}
%%   \city{Hekla}
%%   \country{Iceland}}
%% \email{larst@affiliation.org}

%%
%% By default, the full list of authors will be used in the page
%% headers. Often, this list is too long, and will overlap
%% other information printed in the page headers. This command allows
%% the author to define a more concise list
%% of authors' names for this purpose.
%\renewcommand{\shortauthors}{Trovato et al.}

%%
%% The abstract is a short summary of the work to be presented in the
%% article.
\begin{abstract}
  Step-indexed logical relations are an important tool in the
  metatheory of programming
  languages~\citep{Appel:2001aa,Ahmed:2006aa}, and their application
  can be streamlined by use of modal logics such as
  LSLR~\citep{Dreyer:2011wl}.  The Iris system~\citep{JUNG:2018aa}
  provides support for reasoning about step-indexed logical relations
  in the Coq proof assistant. This paper describes a shallow embedding
  of a step-indexed logic in Agda and its application to a semantic
  type safety proof for an internal language of the Gradually-Typed
  Lambda Calculus.
\end{abstract}

%%
%% The code below is generated by the tool at http://dl.acm.org/ccs.cfm.
%% Please copy and paste the code instead of the example below.
%%
%\begin{CCSXML}
%\end{CCSXML}

%% \ccsdesc[500]{Computer systems organization~Embedded systems}
%% \ccsdesc[300]{Computer systems organization~Redundancy}
%% \ccsdesc{Computer systems organization~Robotics}
%% \ccsdesc[100]{Networks~Network reliability}

%%
%% Keywords. The author(s) should pick words that accurately describe
%% the work being presented. Separate the keywords with commas.
\keywords{step-indexed logical relations, modal logic}

%% \received{20 February 2007}
%% \received[revised]{12 March 2009}
%% \received[accepted]{5 June 2009}

%%
%% This command processes the author and affiliation and title
%% information and builds the first part of the formatted document.
\maketitle

%TODO: Road Map/Outline of what's in the paper

\section{Introduction}

Logical relations are a useful technique for proving properties of
programming languages, from type safety to relational parametricity,
and noninterference.  Initially, logical relations were applied to
terminating languages such as the Simply Typed Lambda
Calculus~\citep{Tait:1967aa} and System F~\citep{GIRARD72}, but
\citet{Appel:2001aa} extended the technique to handle languages with
nontermination by introducing a step index into the logical
relation. After some experience with applications of step-indexed
logical
relations~\citep{Ahmed:2004eu,Ahmed:2009aa,Neis:2009fk,Hur:2011aa},
\citet{Dreyer:2011wl} designed the LSLR logic to streamline the
reasoning about step indexes. This research program eventually
produced Iris~\citep{JUNG:2018aa}, a system that supports a
combination of step indexed logical relations and separation logic
within the Coq proof
assistant~\citep{The-Coq-Development-Team:2004kf,Huet:2016aa}.

The author is interested in step-indexed logical relations for
purposes of developing mechanized proofs in Agda of the Gradual
Guarantee~\citep{Siek:2015ac} and Parametricity~\citep{REYNOLDS74C}
for gradually-typed polymorphic
languages~\citep{Ahmed:2011fk,Ahmed:2017aa,Igarashi:2017aa,New:2019ab,Labrada:2020tk}.
To streamline such developments in Agda, we need support for a
step-indexed logic in Agda. This paper fills that need by showing how
to build a shallow embedding of a step-indexed logic in Agda, a system
that we abbreviate as SIL. To evaluate the utility of SIL, we present
a case study of proving semantic type safety of a simple cast
calculus, i.e., an internal language for the Gradually-Typed Lambda
Calculus (GTLC)~\citep{Siek:2006bh,Siek:2007qy}. We find that SIL
streamlines the definition of logical relations and is helpful in many
of the proofs, but there is room for improvement regarding the
ergonomics of interactive proof development using SIL.  Limitations in
Agda's inference algorithm sometimes leads to verbosity in SIL
proofs. It is worth noting that the authors have also proved the
Gradual Guarantee for the GTLC using SIL and the lesssons from that
experience confirm the findings in this paper.

TODO: ROAD MAP


%\section{A Step-Indexed Logic (SIL)}

%% Figure~\ref{fig:SIL}

%% \begin{figure}
%%   \raggedright
%% Syntax
%%   \begin{syntax}
%%     \text{Propositions} & P,Q &::= & \top \mid \bot \mid P \land Q
%%     \mid P \lor Q \mid P \Rightarrow Q \mid \forall X.P \mid \exists X.P
%%     \mid \overline{M} \in R\\
%%     \text{Relations} & R,S & ::= & r \mid \mu r. R \mid \{ \overline{x} \mid P \}
%%   \end{syntax}
%% Meaning of Propositions
%% \begin{align*}
%%   \sem{ P } \rho\; 0 &= \top \\
%%   \sem{ \top } \rho(suc \app n) &= \top \\
%%   \sem{ \bot } \rho (suc \app n) &= \bot \\
%%   \sem{ P \land Q } \rho (suc \app n) &=
%%     \sem{ P } \rho n \land \sem{ Q } \rho n \\
%%   \sem{ P \lor Q } \rho (suc \app n) &=
%%     \sem{ P } \rho n \lor \sem{ Q } \rho n \\
%%   \sem{ P \Rightarrow Q } \rho (suc \app n) &=
%%     \forall k \leq suc \app n,
%%     \sem{ P } \rho k \implies \sem{ Q } \rho k \\
%%   \sem{ \forall X.P} \rho \app (suc \app n) &= \forall a, \sem{P[a/X]} \rho (suc \app n)\\
%%   \sem{ \exists X.P} \rho \app (suc \app n) &= \exists a, \sem{P[a/X]} \rho (suc \app n)\\
%%   \sem{ \overline{M} \in R } \rho (suc \app n) &=
%%     \sem{ R } \rho (suc \app n) \app \overline{M}\\
%%   \sem{ \rhd P } \rho (suc \app n) &=
%%     \sem{ P } \rho n 
%% \end{align*}
%% Meaning of Relations
%% \begin{align*}
%%   \sem{ R } \rho\, 0 \app \overline{M} &= \top \\
%%   \sem{ r } \rho \app (suc \app n) \app \overline{M} &=
%%     \rho(r) (suc \app n) \app \overline{M} \\
%%   \sem{ \mu r. R } \rho \app (suc \app n)\app \overline{M} &=
%%   F^{suc \app n}(\top) (suc \app n) \app \overline{M} \\
%%   & \text{ where } F(X) = \sem{ R } \rho(r := X) \\
%%   \sem{ \{ \overline{x} \mid P \} } \rho (suc \app n) \app \overline{M} &= 
%%     \sem{ P[\overline{x} := \overline{M}] } \rho (suc \app n)
%% \end{align*}
%%   \caption{Step-Indexed Logic (SIL)}
%%   \label{fig:SIL}
%% \end{figure}


%\section{Case Study: Semantic Type Safety of a Cast Calculus}

%% Figure~\ref{fig:CC}

%% \begin{figure}
%%   \raggedright Syntax
%%   \begin{syntax}
%%     \text{Types} & A,B,C,D &::=& \Bool \mid A \to B \mid \dyn \\
%%     \text{Ground Types} & G,H & ::= & \mathtt{bool} \mid \dyn \to \dyn \\
%%     \text{Booleans} & b & \in & \mathbb{B} \\
%%     \text{Terms} & L,M,N & ::= & x \mid b \mid \lam{x}N \mid L \app M
%%       \mid \inj{M}{G} \mid \proj{M}{H} \mid \mathtt{error} \\
%%       \text{Values} & V,W & ::= & \lam{x}N \mid \inj{V}{G}
%%   \end{syntax}
%%   Type System
%%   \begin{gather*}
%%     \inference{}{\Gamma \vdash x : \Gamma(x)}
%%     \quad
%%     \inference{}{\Gamma \vdash b : \Bool}
%%     \quad
%%     \inference{\Gamma, x : A \vdash N : B}
%%               {\Gamma \vdash \lam{x} N : A \to B}
%%     \quad              
%%     \inference{\Gamma \vdash L : A \to B & \Gamma \vdash M : A}
%%           {\Gamma \vdash L \app M : B} \\[2ex]
%%    \inference{\Gamma \vdash M : G}
%%           {\Gamma \vdash \inj{M}{G} : \dyn}
%%    \quad
%%    \inference{\Gamma \vdash M : \dyn}
%%              {\Gamma \vdash \proj{M}{H} : H}
%%    \quad
%%    \inference{}
%%              {\Gamma \vdash \mathtt{error} : A}
%%   \end{gather*}
%% Reduction
%% \begin{align*}
%%   (\lam{x} N)\app W & \reduce N[x:=W] \\
%%   \proj{\inj{V}{G}}{G} & \reduce V \\
%%   \proj{\inj{V}{G}}{H} & \reduce \texttt{error}
%%        \qquad\qquad\qquad\text{if } G \neq H 
%% \end{align*}

%%   \caption{Internal language (cast calculus) for the Gradually-Typed
%%     Lambda Calculus.}
%%   \label{fig:CC}
%% \end{figure}



%% \begin{align*}
%%     \semE{\_} &: \mathsf{Type} \to \mathsf{Term} \to \mathsf{Set} \\
%%     \semE{A} M &= (\semV{A} M \lor \mathsf{reducible}\ M \lor M=\mathtt{blame})
%%                 \land (\forall N, (M \reduce N) \Rightarrow \rhd \semE{A} N)
%% \end{align*}

\input{latex/cpp2024/StepIndexedLogic.tex}

\bibliographystyle{ACM-Reference-Format}
\bibliography{all}

\end{document}
\endinput

